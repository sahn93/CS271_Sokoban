\documentclass{article}

% ready for submission
\usepackage[final]{neurips_2021}
\usepackage[utf8]{inputenc} % allow utf-8 input
\usepackage[T1]{fontenc}    % use 8-bit T1 fonts
\usepackage{hyperref}       % hyperlinks
\usepackage{url}            % simple URL typesetting
\usepackage{booktabs}       % professional-quality tables
\usepackage{amsfonts}       % blackboard math symbols
\usepackage{nicefrac}       % compact symbols for 1/2, etc.
\usepackage{microtype}      % microtypography
\usepackage{xcolor}         % colors

\title{Sokoban Solver}

\author{
  Seokchan Ahn\\
  Department of Computer Science\\
  University of California, Irvine\\
  Irvine, CA 92697\\
  \texttt{seokchaa@uci.edu} \\
   \And
%   Tharak Sai Bobba \\
%   Department of Computer Science\\
%   University of California, Irvine\\
%   Irvine, CA 92612\\
%   \texttt{tbobba@uci.edu} \\
%   \And
%   Almendra Salgado Gatica \\
%   Department of Computer Science\\
%   University of California, Irvine\\
%   Irvine, CA 92606\\
%   \texttt{arsalgad@uci.edu} \\
}
\begin{document}

\maketitle

\begin{abstract}
  The abstract paragraph should be indented \nicefrac{1}{2}~inch (3~picas) on
  both the left- and right-hand margins. Use 10~point type, with a vertical
  spacing (leading) of 11~points.  The word \textbf{Abstract} must be centered,
  bold, and in point size 12. Two line spaces precede the abstract. The abstract
  must be limited to one paragraph.
\end{abstract}

\section{Submission of Project Abstract}

The assignment is to submit a title and an abstract of your project, i.e. a short and concise description of what you are planning to do.


\appendix

\section{Appendix}

Optionally include extra information (complete proofs, additional experiments and plots) in the appendix.
This section will often be part of the supplemental material.

\end{document}